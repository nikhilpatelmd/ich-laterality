% Options for packages loaded elsewhere
\PassOptionsToPackage{unicode}{hyperref}
\PassOptionsToPackage{hyphens}{url}
\PassOptionsToPackage{dvipsnames,svgnames,x11names}{xcolor}
%
\documentclass[
  letterpaper,
  DIV=11,
  numbers=noendperiod]{scrartcl}

\usepackage{amsmath,amssymb}
\usepackage{iftex}
\ifPDFTeX
  \usepackage[T1]{fontenc}
  \usepackage[utf8]{inputenc}
  \usepackage{textcomp} % provide euro and other symbols
\else % if luatex or xetex
  \usepackage{unicode-math}
  \defaultfontfeatures{Scale=MatchLowercase}
  \defaultfontfeatures[\rmfamily]{Ligatures=TeX,Scale=1}
\fi
\usepackage{lmodern}
\ifPDFTeX\else  
    % xetex/luatex font selection
\fi
% Use upquote if available, for straight quotes in verbatim environments
\IfFileExists{upquote.sty}{\usepackage{upquote}}{}
\IfFileExists{microtype.sty}{% use microtype if available
  \usepackage[]{microtype}
  \UseMicrotypeSet[protrusion]{basicmath} % disable protrusion for tt fonts
}{}
\makeatletter
\@ifundefined{KOMAClassName}{% if non-KOMA class
  \IfFileExists{parskip.sty}{%
    \usepackage{parskip}
  }{% else
    \setlength{\parindent}{0pt}
    \setlength{\parskip}{6pt plus 2pt minus 1pt}}
}{% if KOMA class
  \KOMAoptions{parskip=half}}
\makeatother
\usepackage{xcolor}
\setlength{\emergencystretch}{3em} % prevent overfull lines
\setcounter{secnumdepth}{-\maxdimen} % remove section numbering
% Make \paragraph and \subparagraph free-standing
\ifx\paragraph\undefined\else
  \let\oldparagraph\paragraph
  \renewcommand{\paragraph}[1]{\oldparagraph{#1}\mbox{}}
\fi
\ifx\subparagraph\undefined\else
  \let\oldsubparagraph\subparagraph
  \renewcommand{\subparagraph}[1]{\oldsubparagraph{#1}\mbox{}}
\fi


\providecommand{\tightlist}{%
  \setlength{\itemsep}{0pt}\setlength{\parskip}{0pt}}\usepackage{longtable,booktabs,array}
\usepackage{calc} % for calculating minipage widths
% Correct order of tables after \paragraph or \subparagraph
\usepackage{etoolbox}
\makeatletter
\patchcmd\longtable{\par}{\if@noskipsec\mbox{}\fi\par}{}{}
\makeatother
% Allow footnotes in longtable head/foot
\IfFileExists{footnotehyper.sty}{\usepackage{footnotehyper}}{\usepackage{footnote}}
\makesavenoteenv{longtable}
\usepackage{graphicx}
\makeatletter
\def\maxwidth{\ifdim\Gin@nat@width>\linewidth\linewidth\else\Gin@nat@width\fi}
\def\maxheight{\ifdim\Gin@nat@height>\textheight\textheight\else\Gin@nat@height\fi}
\makeatother
% Scale images if necessary, so that they will not overflow the page
% margins by default, and it is still possible to overwrite the defaults
% using explicit options in \includegraphics[width, height, ...]{}
\setkeys{Gin}{width=\maxwidth,height=\maxheight,keepaspectratio}
% Set default figure placement to htbp
\makeatletter
\def\fps@figure{htbp}
\makeatother

\KOMAoption{captions}{tableheading}
\makeatletter
\makeatother
\makeatletter
\makeatother
\makeatletter
\@ifpackageloaded{caption}{}{\usepackage{caption}}
\AtBeginDocument{%
\ifdefined\contentsname
  \renewcommand*\contentsname{Table of contents}
\else
  \newcommand\contentsname{Table of contents}
\fi
\ifdefined\listfigurename
  \renewcommand*\listfigurename{List of Figures}
\else
  \newcommand\listfigurename{List of Figures}
\fi
\ifdefined\listtablename
  \renewcommand*\listtablename{List of Tables}
\else
  \newcommand\listtablename{List of Tables}
\fi
\ifdefined\figurename
  \renewcommand*\figurename{Figure}
\else
  \newcommand\figurename{Figure}
\fi
\ifdefined\tablename
  \renewcommand*\tablename{Table}
\else
  \newcommand\tablename{Table}
\fi
}
\@ifpackageloaded{float}{}{\usepackage{float}}
\floatstyle{ruled}
\@ifundefined{c@chapter}{\newfloat{codelisting}{h}{lop}}{\newfloat{codelisting}{h}{lop}[chapter]}
\floatname{codelisting}{Listing}
\newcommand*\listoflistings{\listof{codelisting}{List of Listings}}
\makeatother
\makeatletter
\@ifpackageloaded{caption}{}{\usepackage{caption}}
\@ifpackageloaded{subcaption}{}{\usepackage{subcaption}}
\makeatother
\makeatletter
\@ifpackageloaded{tcolorbox}{}{\usepackage[skins,breakable]{tcolorbox}}
\makeatother
\makeatletter
\@ifundefined{shadecolor}{\definecolor{shadecolor}{rgb}{.97, .97, .97}}
\makeatother
\makeatletter
\makeatother
\makeatletter
\makeatother
\ifLuaTeX
  \usepackage{selnolig}  % disable illegal ligatures
\fi
\IfFileExists{bookmark.sty}{\usepackage{bookmark}}{\usepackage{hyperref}}
\IfFileExists{xurl.sty}{\usepackage{xurl}}{} % add URL line breaks if available
\urlstyle{same} % disable monospaced font for URLs
\hypersetup{
  pdftitle={Hemispheric Lateralization in ICH, Functional Disability, and Health-Related Quality of Life},
  colorlinks=true,
  linkcolor={blue},
  filecolor={Maroon},
  citecolor={Blue},
  urlcolor={Blue},
  pdfcreator={LaTeX via pandoc}}

\title{Hemispheric Lateralization in ICH, Functional Disability, and
Health-Related Quality of Life}
\author{}
\date{}

\begin{document}
\maketitle
\ifdefined\Shaded\renewenvironment{Shaded}{\begin{tcolorbox}[sharp corners, frame hidden, interior hidden, borderline west={3pt}{0pt}{shadecolor}, enhanced, breakable, boxrule=0pt]}{\end{tcolorbox}}\fi

\hypertarget{protocol-version-1.0}{%
\subsection{Protocol Version 1.0}\label{protocol-version-1.0}}

\textbf{Investigators}: Nikhil Patel, Rahul Karamchandani, Andrew Asimos

\textbf{Sponsor or funding source}: none

\textbf{Background, Rationale, and Context}: Intracerebral hemorrhage
(ICH) is a devastating type of stroke that results from bleeding within
the brain tissue. It is a significant cause of morbidity and mortality,
with high rates of disability and death. Hemispheric lateralization,
which refers to the specialization of the left and right hemispheres of
the brain for specific functions, has been suggested to play a role in
the functional outcomes of patients with ICH. The left hemisphere is
generally associated with language, verbal memory, and logical
reasoning, while the right hemisphere is associated with spatial
processing, attention, and emotional regulation.

Several studies have investigated the relationship between hemispheric
lateralization and functional outcomes in patients with ICH, with
conflicting results. Some studies have found that left hemisphere ICH is
associated with worse functional outcomes than right hemisphere ICH,
while others have found the opposite.

Understanding the relationship between hemispheric lateralization and
functional outcomes in patients with ICH is essential for improving
patient care and developing targeted rehabilitation strategies.
Therefore, this paper aims to pool individual patient data from several
prospective studies funded by the National Institute of Neurological
Disorders and Stroke (NINDS), a division of the National Institute of
Health (NIH), to investigate the association of hemispheric
lateralization on outcomes. This data is publicly available upon
submitting on application to the NINDS and contains anonymized
individual patient data.

\textbf{Objectives}: The primary objective of this research project is
to determine any association between hemispheric laterality and
functional outcomes in patients with ICH. Secondary objectives are to
investigate any association between hemispheric laterality and
aggressiveness of care as well as any association between hemispheric
laterality and health-related quality of life.

\textbf{Study Type}: Retrospective analysis of anonymized individual
patient data from NINDS observational studies and randomized clinical
trials

\textbf{Selection criteria}:

\begin{itemize}
\item
  \textbf{Inclusion}:

  \begin{itemize}
  \tightlist
  \item
    Enrolled in the following NINDS studies: ERICH, ATACH-2, iDEF,
    MISTIE, MISTIE-III, and CLEAR-III
  \item
    At least 18 years of age
  \item
    ICH location in the following areas: lobar, basal ganglia, or
    thalamus
  \end{itemize}
\item
  \textbf{Exclusion}:

  \begin{itemize}
  \tightlist
  \item
    ICH location in the brainstem or cerebellum
  \item
    Primary intraventricular hemorrhage (IVH)
  \item
    Midline location of hemorrhage
  \end{itemize}
\end{itemize}

\textbf{Time period}: 2017 - 2013

\textbf{Outcomes}:

\begin{itemize}
\tightlist
\item
  \textbf{Primary}: modified Rankin Scale at 90 days
\item
  \textbf{Secondary}:

  \begin{itemize}
  \tightlist
  \item
    Percentage of patients undergoing neurosurgical intervention
  \item
    Percentage of patients with early withdrawal of life-sustaining
    therapy
  \item
    Percentage of patients with tracheostomy
  \item
    EuroQOL Visual Analog Score (VAS) at 90 days
  \item
    Barthel Index at 90 days
  \end{itemize}
\end{itemize}

\textbf{Definitions}:

\begin{itemize}
\tightlist
\item
  Neurosurgical intervention: any patient undergoing hematoma
  evacuation, external ventricular drain, or decompressive craniectomy
\item
  Early withdrawal of life-sustaining therapy: any patient with code
  status change to DNR or DNI as well as any patient transitioned to
  comfort-measures only within 72 hours of admission
\end{itemize}

\textbf{Data collection}

\begin{itemize}
\item
  \textbf{Baseline characteristics}:

  \begin{itemize}
  \tightlist
  \item
    Age
  \item
    Gender
  \item
    Race
  \item
    Ethnicity
  \item
    Admission Glasgow Coma Score (GCS)
  \item
    ICH Volume (mL)
  \item
    Presence of IVH
  \item
    ICH Location
  \item
    Admission Systolic Blood Pressure
  \item
    Admission Diastolic Blood Pressure
  \item
    Comorbidities:

    \begin{itemize}
    \tightlist
    \item
      Coronary artery disease
    \item
      Congestive heart failure
    \item
      Atrial fibrillation
    \item
      Chronic kidney disease
    \item
      Previous stroke
    \item
      Tobacco use
    \item
      Alcohol use
    \item
      Recreational drug use
    \end{itemize}
  \end{itemize}
\item
  \textbf{Inpatient care characteristics}:

  \begin{itemize}
  \tightlist
  \item
    Neurosurgical intervention
  \item
    Early withdrawal of life-sustaining therapies
  \item
    Tracheostomy
  \item
    Gastric tube
  \end{itemize}
\item
  \textbf{Outcomes}:

  \begin{itemize}
  \tightlist
  \item
    modified Rankin scale at 90 days
  \item
    EuroQOL VAS at 90 days
  \item
    Barthel Index at 90 days
  \end{itemize}
\end{itemize}

\textbf{Analytical Plan}: Results will be analyzed initially using
descriptive statistics with means and standard deviations or counts and
percentages. Ordinal regression analysis will be utilized for the
primary outcome and logistic regression for secondary outcomes. Other
inferential statistical analysis will be conducted as appropriate. Alpha
will be set at 0.05 to be considered statistically significant.

\textbf{Human Subjects Protection}:

\begin{itemize}
\item
  \textbf{Informed Consent}: Written informed consent will not be
  obtained. The requested data is publicly available and contains
  anonymized patient data that cannot be traced back to the original
  patient.
\item
  \textbf{Confidentiality and Privacy}: Patient data is already
  anonymized and cannot be traced back to the original patient. All
  terms agreed upon by signing the NINDS Data Request Agreement will be
  followed.
\item
  \textbf{Data and Safety Monitoring}: The principal investigator will
  be responsible for the overall monitoring of the data and safety of
  study participants. The principal investigator will be assisted by
  other members of the study staff. All terms agreed upon by signing the
  NINDS Data Request Agreement will be followed.
\end{itemize}



\end{document}
